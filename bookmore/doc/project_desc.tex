% Bookmore (more than a bookmark) is a web application to management bookmarks.
% Copyright (C) 2012 Román Ginés Martínez Ferrández <rgmf@riseup.net>
%
% This is the project documentatio that is free software: you can redistribute it and/or modify
% it under the terms of the GNU General Public License as published by
% the Free Software Foundation, either version 3 of the License, or
% (at your option) any later version.
%
% This program is distributed in the hope that it will be useful,
% but WITHOUT ANY WARRANTY; without even the implied warranty of
% MERCHANTABILITY or FITNESS FOR A PARTICULAR PURPOSE.  See the
% GNU General Public License for more details.
%
% You should have received a copy of the GNU General Public License
% along with this program.  If not, see <http://www.gnu.org/licenses/>.

\documentclass[12pt,twoside,a4paper]{article}

\usepackage[spanish,activeacute]{babel}
\usepackage[utf8]{inputenc}
\usepackage{fancyhdr} % Para modificar y gestionar los encabezado y pie de página.
\usepackage[top=3cm, bottom=3cm, left=2cm, right=2cm]{geometry} % De esta forma se indican los margenes del papel.
\usepackage{url}

\pagestyle{fancy} % Usamos el estilo ``fancy'' para encabezado y pie de página.
\fancyhead{} % Reseteamos encabezado.
\fancyfoot{} % Reseteamos pie de página.

% Configuramos los encabezado y pie de página. Significados de las letras:
% L: izquierda.
% R: derecha.
% C: centro.
% O: par.
% E: impar.
\fancyhead[LO,LE]{\slshape \rightmark}
\fancyhead[RO,RE]{\slshape \leftmark}
\fancyfoot[CO,CE] {\thepage}

% Define la línea de debajo y encima del encabezado y pie de página: 0pt = invisible.
\renewcommand{\headrulewidth}{0.4pt}
\renewcommand{\footrulewidth}{0.4pt}

\title{Proyecto \texttt{Bookmore}}
\author{Román Ginés Martínez Ferrández \\ \texttt{rgmf@riseup.net}}

% BEGIN DOCUMENT
\begin{document}
\maketitle
\thispagestyle{fancy} % Con esto en la portada también aparecer encabezado y pie de página. Si no deseas en la portada encabezado y pie de página elimina esta línea.
\tableofcontents

% Empieza aquí
\section{Arquitectura}
Para este subproyecto se va a emplear el patrón de diseño MVC (Model View Controller) con el framework \texttt{Yii Framework} (\url{http://www.yiiframework.com}).

\subsection{Requerimientos}
Se necesita tener instalado:
\begin{itemize}
\item MySQL.
\item Apache con el módulo ``rewrite'' (libapache2-mod-proxy-html - Apache2 filter module for HTML links rewriting)
\item PHP.
\item Yii Framework 1.1.x.
\end{itemize}

En concreto esta aplicación web es desarrollada con Apache2, MySql 5.1.63, PHP 5.3.6 y Yii Framework 1.1.10.

\subsection{Criterios de codificación}
Los criterios de codificación que se van a seguir en este proyecto son los mismos que los empleados en Yii Framework.

\subsection{Estructura de directorios}
La estructura de directorios utilizada es la creada por la herramienta \texttt{yiic} que viene con \texttt{Yii Framework}. Además, se pueden añadir otros directorios como, por ejemplo, el directorio que contiene esta documentación (\texttt{doc}).

\section{Proyectos}
Esta aplicación web debe ofrecer la posibilidad de recopilar información de recursos en general: páginas web favoritas (pueden ser sitios completos o páginas concretas dentro de un sitio web), imágenes (que estarán en el servidor web alojadas), documentos (pdfs, textos, etc - que estarán alojadas en el servidor web).

De las páginas webs se necesita recopilar: dirección web (URL), tag o tags, una breve descripción, el logotipo de la página web (que puede ayudar para identificarla), la fecha en que se creó el marcador. Además, asociada a cada página web, se puede guardar información acerca del usuario y la contraseña si se tiene usuario en dicha página web.

De los recursos (documentos, imágenes, etc) se necesita: el enlace al archivo, el nombre del archivo, el formato, una descripción, un tag o tags y la fecha en que fue dada de alta el recurso.

El usuario que crea el recurso puede determinar si el mismo es privado o público. Por defecto los marcadores serán públicos y el resto de recursos privados. El usuario y contraseña de aquellas páginas web donde se haya registrado será privada.

Los usuarios anónimos pueden ver la información de todos los recursos públicos, con sus descripciones, y las pueden valorar. Cada usuario registrado puede añadir y eliminar sus páginas webs, así como ver su nombre de usuario y contraseña de aquellas páginas donde está registrado, y, por último, también pueden valorar los recursos (públicos y de los que es usuario) y comentarlos.

\end{document}
