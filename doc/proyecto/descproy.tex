% Utamaduni (suajili language meaning 'culture') is a book management.
% Copyright (C) 2012 Román Ginés Martínez Ferrández <rgmf@riseup.net>
%
% This is the project documentatio that is free software: you can redistribute it and/or modify
% it under the terms of the GNU General Public License as published by
% the Free Software Foundation, either version 3 of the License, or
% (at your option) any later version.
%
% This program is distributed in the hope that it will be useful,
% but WITHOUT ANY WARRANTY; without even the implied warranty of
% MERCHANTABILITY or FITNESS FOR A PARTICULAR PURPOSE.  See the
% GNU General Public License for more details.
%
% You should have received a copy of the GNU General Public License
% along with this program.  If not, see <http://www.gnu.org/licenses/>.

\documentclass[12pt,twoside,a4paper]{article}

\usepackage[spanish,activeacute]{babel}
\usepackage[utf8]{inputenc}
\usepackage{fancyhdr} % Para modificar y gestionar los encabezado y pie de página.
\usepackage[top=3cm, bottom=3cm, left=2cm, right=2cm]{geometry} % De esta forma se indican los margenes del papel.

\pagestyle{fancy} % Usamos el estilo ``fancy'' para encabezado y pie de página.
\fancyhead{} % Reseteamos encabezado.
\fancyfoot{} % Reseteamos pie de página.

% Configuramos los encabezado y pie de página. Significados de las letras:
% L: izquierda.
% R: derecha.
% C: centro.
% O: par.
% E: impar.
\fancyhead[LO,LE]{\slshape \rightmark}
\fancyhead[RO,RE]{\slshape \leftmark}
\fancyfoot[CO,CE] {\thepage}

% Define la línea de debajo y encima del encabezado y pie de página: 0pt = invisible.
\renewcommand{\headrulewidth}{0.4pt}
\renewcommand{\footrulewidth}{0.4pt}

\title{Proyecto \texttt{Machafuko}}
\author{Román Ginés Martínez Ferrández \\ \texttt{romangines@riseup.net}}

% BEGIN DOCUMENT
\begin{document}
\maketitle
\thispagestyle{fancy} % Con esto en la portada también aparecer encabezado y pie de página. Si no deseas en la portada encabezado y pie de página elimina esta línea.
\tableofcontents

% Empieza aquí
\section{Arquitectura}
Este proyecto es, en realidad, un conjunto de proyectos que pretenden gestionar una serie de necesidades surgidas. De ahí el nombre de \emph{machafuko}, que en \emph{suajili} significa ``caos''. Los proyectos de que constan son:
\begin{itemize}
\item \textbf{utamaduni}, que significa ``cultura'' en \emph{suajili}.
\item \textbf{ajenda}, que significa ``agenda'' en \emph{suajili}.
\item \textbf{tarajali}, que significa ``apendriz'' en \emph{suajili}.
\end{itemize}
Para todos estos subproyectos se va a emplear el patrón de diseño MVC (Model View Controller).

\subsection{Criterios de codificación}
A continuación se especifican los criterios que se deben seguir para codificar este programa. Dado que no hay unanimidad en PHP para la codificación del código, se emplearán unos criterios parecidos a los seguidos en la programación C/C++.
\begin{itemize}
\item Los nombres de los identificadores deben ser claros, cortos e informativos.
\item Se separarán palabras con el carácter ``\_'' para todo tipo de identificadores (variables, constantes, nombres de función, nombres de clase, etc).
\item Se usarán minúsculas en los nombres de variables, funciones, métodos y clases.
\item Las variables globales comenzarán por el prefijo ``g\_'' (aunque se debe evitar su uso).
\item Las constantes comenzarán con el carácter ``k'' (minúscula) seguido por el identificador en mayúsculas.
\item El nombre de las clases que pertenezcan al controlador acabarán con el sufijo ``\_controller''.
\item El nombre de las clases que pertenezcan al modelo acabarán con el sufijo ``\_model''.
\end{itemize}

\subsection{Estructura de directorios}
El proyecto consta de un fichero ``index.html'' que nos llevará a cada uno de los subproyectos mencionados y que tendrán, todos ellos, la siguiente estructura de directorios:
\begin{itemize}
\item \textbf{doc}: en este directorio se almacenará toda la documentación del proyecto.
\item \textbf{po}: los archivos de localización e internacionalización.
\item \textbf{db}: todos los archivos relacionados con la base de datos (backups, sql, etc).
\item \textbf{log}: los logs que genere la aplicación (sobre todo para la depuración).
\item \textbf{tmp}: para que la aplicación pueda almacenar archivos temporales.
\item \textbf{app}: aquí está la aplicación como tal.
  \begin{itemize}
  \item \textbf{include}: aquí dentro está el ``framework'', ``helpers'', ``utils'', ``decorator'', etc. Es decir, todo aquello considerado el ``núcleo'' del sistema.
  \item \textbf{site}: directorio exclusivo para los diseñadore, es decir, la parte estática del sistema.
    \begin{itemize}
    \item \textbf{css}: los estilos.
    \item \textbf{img}: las imágenes.
    \item \textbf{js}: el JavaScript.
    \item \textbf{html}: ficheros HTML generales.
      \begin{itemize}
      \item \textbf{tpl}: plantillas HTML generales.
      \end{itemize}
    \end{itemize}
  \item \textbf{modules}: los módulos de la aplicación (el modelo). Cada modelo tiene su directorio \textbf{tpl} con la/s página/s o template/s.
  \end{itemize}
\end{itemize}

\section{Proyectos}
Las diferentes partes de que consta la aplicación y su descripción se detallan en los siguientes apartados.

\subsection{Utamaduni}
\subsubsection{Libros}
Esta parte de la aplicación permitirá gestionar la información de los libros que se leen y se requieren:
\begin{itemize}
\item ISBN.
\item Título.
\item Autor (o autores).
\item Género (o géneros) en forma de etiquetas.
\item Fechas de compra, inicio de lectura y fin de lectura.
\item Descripción.
\item Valoración numérica.
\item Comentario tras haber leído el libro.
\item Formato (ebook o papel).
\end{itemize}
Además, interesa saber si el libro lo tengo yo en posesión (que me costó en este caso), si me lo han dejado (quién me lo ha dejado) o si lo he sacado de una biblioteca (qué biblioteca).

\subsubsection{Frases célebres}
Interesa almacenar frases célebres que pueden haber sido obtenidas de películas, libros, Internet o me las han pasado de alguna forma. En cualquier caso se almacenará de dichas frases célebres:
\begin{itemize}
\item Autor (puede ser anónimo).
\item Frase en sí.
\end{itemize}
Si ha sido sacada de alguno de los libros leídos se deberá indicar el libro y la página donde está esa frase o parte del libro.

\subsection{Ajenda}
\subsubsection{Contactos}
En esta aplicación también se pueden almacenar contactos de los que se requiere:
\begin{itemize}
\item Nombre y apellidos.
\item Apodo o nickname.
\item E-mail (o e-mails).
\item Fecha de nacimiento.
\item Dirección postal.
\item Grupo/s (o etiqueta/s) - amistad, familiar, compañero, conocido...
\end{itemize}
Los contactos pueden ser importados/exportados en formato CSV (al menos) indicando carácter separador.

\subsubsection{Calendario}
Se pueden tener varios calendarios (laboral y personal, por ejemplo). En el calendario se pueden anotar todo tipo de tareas a hacer:
\begin{itemize}
\item Trabajo.
\item Citas (de todo tipo).
\item Proyectos.
\item Libros por leer.
\item Películas por ver.
\item Retos que conseguir.
\item Recordatorios.
\item Etc.
\end{itemize}
Las tareas se podrán ver de muchas maneras, por ejemplo: por grupos, tareas realizadas o acabadas, tareas para hoy, tareas para mañana, tareas para pasado mañana, tareas para esta semana, tareas para este mes y tareas para este año.
Se pueden programar recordatorios que serán enviados por e-mail a una o varias cuentas indicadas.
Los calendarios pueden ser importados/exportados en formato CSV, indicando carácter separador.
De esta forma por cada entrada en el calendario se almacenará:
\begin{itemize}
\item Fecha y horas de inicio y de fin.
\item Descripción de la tarea, recordatorio, etc.
\item Etiqueta (trabajo, libro, película, reto, recordatorio...).
\item Fecha y hora de recordatorio.
\item E-mail o e-mails para enviar el recordatorio por e-mail.
\end{itemize}

\subsection{Tarajali}
\subsubsection{Trabajo}
En el apartado de trabajo existen tres perfiles bien diferenciados que se pasan a describir en los siguientes apartados.

\begin{itemize}
\item \textbf{Administrador}.\\
Puede crear/eliminar profesores/alumnado. Además será siempre un profesor. Es decir, el administrador será un profesor con permisos de administración y será configurado al inicio (al instalar la aplicación).

\item \textbf{Profesor}.\\
De los profesores se guardará:
\begin{itemize}
\item Nombre y apellidos.
\item E-mail.
\item Usuario para el acceso al sistema.
\item Contraseña para el acceso al sistema.
\item Si es o no el administrador.
\end{itemize}
Cada profesor puede:
\begin{itemize}
\item Gestionar cursos y asignaturas. ¡OJO!, si se elimina un curso se eliminará también todo el alumnado de dicho curso, con todas sus notas, etc.
\item Gestionar el alumnado.
\item Gestionar notas (conceptos, procedimientos y actitudes).
\item Gestionar exámenes, ejercicios, ejercicios online y prácticas.
\item Comunicarse con uno, varios o todo el alumnado de un curso.
\item Compartir material con uno, varios o todo el alumnado de un curso.
\item Corregir ejercicios/exámenes online dejando anotaciones por cada pregunta.
\item Dejar una nota sobre las prácticas corregidas.
\end{itemize}

\item \textbf{Alumnado}.\\
De los estudiantes se almacenará:
\begin{itemize}
\item Nombre y apellidos.
\item E-mail.
\item Usuario para el acceso al sistema.
\item Contraseña para el acceso al sistema.
\item Asignatura o asignaturas a las que está matriculado con dicho profesor.
\item Sitio en que se sienta.
\end{itemize}
El estudiante puede realizar las siguientes acciones:
\begin{itemize}
\item Ver su información personal y cambiarla.
\item Ver sus notas (ejercicios, exámenes y observaciones).
\item Publicar entradas en su blog que podrá ver el profesor y sus compañeros.
\item Comunicarse con el profesor a través de notas.
\item Realizar ejercicios, exámenes y prácticas online.
\item Descargarse los enunciados de los ejercicios, exámenes y prácticas.
\item Ver los ejercicios/exámenes online corregidos con sus anotaciones.
\item Ver la nota de las prácticas corregidas por el profesor.
\item Ver las entregas que ha realizado en cada práctica.
\end{itemize}
\end{itemize}

\end{document}
